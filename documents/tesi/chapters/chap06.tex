\chapter{Conclusioni}
\section{Considerazioni finali}
In conclusione, la creazione di un \textit{serious game} per la gestione e l'esecuzione di quiz in ambito educativo si è rivelata un'idea promettente e realizzabile.

Per testare l'efficacia del gioco, è stato possibile condurre un breve test in una classe di scuola media.
Durante questa prova, gli studenti hanno potuto giocare a dei quiz creati dalla loro insegnante.

Il feedback degli studenti è stato generalmente positivo. Soprattutto è stata osservata una buona motivazione dovuta da una spontanea competizione tra amici.

Anche considerando le limitazioni del tesi, il risultato è stato soddisfacente e incoraggiante per futuri sviluppi.

\section{Limitazioni}
Purtroppo il test condotto è stato molto limitato, sia per il numero di studenti e insegnanti coinvolti, sia per la sua durata.
Questo non permette di essere sicuri dell'efficacia del gioco in un contesto scolastico più ampio, per cui bisognerebbe condurre altre prove.

Dal punto di vista tecnico, la limitazione principale è la mancanza di varietà nei tipi di domande supportate. Attualmente, il gioco supporta solo domande a risposta multipla, anche se con la possibilità di avere un numero variabile di risposte corrette.

\section{Sviluppi futuri}
Come accennato nelle limitazioni, dei possibili sviluppi futuri possono riguardare l'aggiunta di diversi tipi di domande, come domande a vero/falso, oppure di modalità di gioco alternative.

La grafica è stata scelta per avvicinarsi al tema della storia, ma questo non implica che i quiz debbano essere necessariamente sulla storia.

Un'altra possibile espansione futura potrebbe quindi essere l'aggiunta di diverse categorie tematiche, cioè diverse materie scolastiche , che mantengano la stessa struttura e le stesse funzionalità attuali, ma un design grafico adatto alla materia scelta.
Questo renderebbe il gioco più interessante, aumentando la varietà grafica senza dover modificare il funzionamento del gioco.


\newpage

\section{Ringraziamenti}
Desidero esprimere la mia gratitudine a tutti coloro che hanno contribuito alla realizzazione di questo progetto. In particolare, ringrazio il mio relatore, il Prof. Gargantini Angelo Michele, per il suo supporto e la sua guida durante tutto il percorso di sviluppo. Un ringraziamento speciale va anche alla Prof.ssa Bonfanti Silvia per i suoi preziosi consigli e il suo incoraggiamento.
Infine, ringrazio la mia famiglia e i miei amici per il loro sostegno costante e la loro pazienza durante le fasi più impegnative del progetto.
