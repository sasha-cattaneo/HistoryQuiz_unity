\chapter{Introduzione}
Il panorama educativo contemporaneo, sia in Italia che nel resto del mondo, sta attraversando una fase di profonda trasformazione, spinta dalla progressiva digitalizzazione dei processi di apprendimento.

Tra questi nuovi strumenti tecnologici, una delle nuove soluzioni utilizzate è quella dei \textit{serious games}, cioè applicazioni digitali create per unire dinamiche ludiche con obbiettivi educativi, bilanciando elementi 
per intrattenere e coinvolgere il giocatore al rigore didattico della scuola.

\bigskip

La presente tesi si pone l’obiettivo di esplorare le potenzialità di tali strumenti
applicati all’insegnamento della storia contemporanea, con particolare riferimento agli eventi del Novecento.
Questa disciplina, sebbene fondamentale per la comprensione delle dinamiche socio-politiche odierne, 
viene spesso trascurata dagli studenti e sacrificata dagli insegnanti, che spesso devono omettere, o trattare frettolosamente, gli ultimi decenni del Novecento per mancanza di tempo.

Lo scopo centrale di questo progetto di tesi è di sviluppare Quiz me!, un  \textit{serious game}, progettato come strumento di supporto didattico, integrativo alla lezione tradizionale.
Questo gioco mira ad aiutare gli insegnanti a completare lo studio della storia fino all’attualità, nonostante la vastità del programma, e ad aumentare l’interesse degli studenti nella materia.

\section{Motivazione e obiettivi}
Attualmente lo studio della storia è considerato dagli studenti come uno sforzo troppo pesante comparato alla sua percepita importanza. Dal report di GoStudent.org, solo il 7\% degli studenti intervistati ha scelto la storia come la propria materia preferita [1].
Questo dato evidenza una mancanza di interesse verso la materia, che può essere attribuita a diversi fattori, tra cui l’approccio didattico tradizionale, spesso percepito come noioso e poco coinvolgente.

% Inoltre la vastità del programma rende impossibile lo studio dettagliato di ogni avvenimento storico, per questo la parte finale del programma deve spesso venire sacrificata, nell’interesse della chiarezza di ciò che si riesce a coprire.

Come evidenziato da Liliana Segre, \textit{“I ragazzi oggi non hanno adeguati strumenti culturali e storici”} (Liliana Segre, 2025)[2], questo impedisce ai giovani di sviluppare lo spirito critico necessario per navigare la complessità dell’attuale mondo socio-politico. Senza una adeguata conoscenza del proprio passato recente si rischia di ripetere gli stessi errori dei nostri antenati.

\bigskip

Il progetto Quiz me! si prefigge i seguenti obiettivi principali:
\begin{itemize}
    \item \textbf{Supporto alla Didattica:} Fornire ai docenti uno strumento versatile per coprire i moduli del Novecento in modo dinamico, permettendo di recuperare il gap temporale tra il programma svolto e l'attualità.
    \item \textbf{Aumento dell'Engagement:} Utilizzare meccaniche di \textit{gamification} (come punteggi, livelli e feedback immediato) per trasformare lo studio mnemonico in una sfida stimolante.
    \item \textbf{Accessibilità e Flessibilità:} Sviluppare una piattaforma intuitiva che permetta allo studente di ripassare i contenuti in autonomia, favorendo l'apprendimento auto-diretto.
\end{itemize}

In ambito ingegneristico, la sfida consiste nel progettare un'architettura software che garantisca un'esperienza utente fluida e facile da usare, in modo di non compromettere il fine educativo. Allo stesso tempo deve essere sia leggera, scalabile e compatibile con diversi dispositivi, per massimizzare l'accessibilità.
