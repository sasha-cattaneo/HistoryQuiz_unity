\chapter{Introduzione}
Negli ultimi anni si è assistito ad una progressiva digitalizzazione nei metodi di 
apprendimento implementati nelle scuole italiane. Una delle nuove tecnologie utilizzate 
è quella dei \textit{serious games}, cioè giochi creati con un fine educativo mantenendo però elementi 
per intrattenere e coinvolgere il giocatore.

\bigskip

La presente tesi nasce dalla volontà di applicare questi nuovi strumenti alla storia, 
soprattutto del Novecento. Una materia che è fondamentale, ma spesso trascurata sia da 
studenti sia da insegnanti, che spesso devono sacrificare gli ultimi decenni del Novecento 
per mancanza di tempo.

Lo scopo centrale di questo progetto è di sviluppare \textcolor{red}{NOME}, un  \textit{serious game}, 
per aiutare gli insegnanti a completare lo studio della storia fino all’attualità, nonostante la vastità del programma, e per aumentare l’interesse degli studenti nella materia.

\section{Motivazione e obiettivi}
Attualmente lo studio della storia è considerato dagli studenti come uno sforzo troppo pesante comparato alla sua percepita importanza. Dal report di GoStudent.org, solo il 7\% degli studenti intervistati ha scelto la storia come la propria materia preferita [1].

Inoltre la vastità del programma rende impossibile lo studio dettagliato di ogni avvenimento storico, per questo la parte finale del programma deve spesso venire sacrificata, nell’interesse della chiarezza di ciò che si riesce a coprire.

Il risultato di ciò è che \textit{“I ragazzi oggi non hanno adeguati strumenti culturali e storici”} (Liliana Segre, 2025)[2] necessari per navigare l’attuale mondo socio-politico e per evitare di ripetere gli stessi errori dei nostri antenati.
