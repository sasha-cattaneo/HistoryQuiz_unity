\chapter{Strumenti utilizzati}
Per sviluppare efficientemente il Serious Game Quiz me! è stato necessario selezionare tecnologie attuali, capaci di soddisfare i requisiti del progetto e in grado di garantire robustezza e scalabilità dell'applicazione.

Questo approccio è stato adottato sia durante le fasi di progettazione sia durante lo sviluppo del gioco.

\section{Strumenti di progettazione}
Per aiutare nella gestione del lavoro, durante ogni fase del progetto, è stata utilizzata Trello [8], una piattaforma online per la creazione e mantenimento di \textit{Kanban board}. Le \textit{Kanban board} sono state utilizzate per tenere traccia delle attività da svolgere, di quelle in corso e di quelle completate, permettendo una gestione visiva e dinamica del progresso del lavoro.

Per la creazione dei vari diagrammi UML è stato utilizzato il sito web diagrams.net, scelto per la completezza dello strumento unita alla sua facilità di utilizzo.

Invece, per creare il diagramma di Gantt è stato utilizzato il programma gratuito e multipiattaforma GanttProject [9].

\section{Strumenti di sviluppo}

\subsection{Game engine e linguaggio di programmazione}
Unity [10] è stato scelto come \textit{game engine} su cui sviluppare Quiz me!. Questa scelta è stata fatta prendendo in considerazione l’accessibilità e, soprattutto, la facilità di utilizzo, la completezza e la compatibilità del motore di gioco.

Avendo scelto Unity, come \textit{game engine}, il linguaggio di programmazione utilizzato nello sviluppo di Quiz me! è stato C\#.

C\# è un linguaggio di programmazione moderno, che ha permesso di sviluppare l’intera logica del gioco, dalla gestione delle scene alla gestione dei file utilizzati come quiz.
Grazie ad esso, in combinazione con Unity, possiamo garantire che Quiz me! sia stabile e compatibile su diverse piattaforme, come Windows, Mac e Linux.

\subsection{Editor di testo e controllo di versione}
Unity non ha un editor di testo incorporato, quindi è stato utilizzato Visual Studio Code [11] per scrivere il codice, che è stato poi incorporato nel gioco tramite Unity Editor, strumento che permette un’intuitiva gestione della grafica utilizzata nel progetto.

Come controllo di versione del gioco è stato utilizzato Git [12], grazie al quale è stato possibile mantenere l’integrità del codice, avere una registrazione dell’evoluzione del gioco e soprattutto salvaguardare il progresso svolto anche in caso di incidenti o errori. 
Per permettere di utilizzare Git su diversi computer, e come forma di \textit{backup} del progetto, è stata scelta la piattaforma gratuita GitHub [13]. La scelta di GitHub è stata fatta in quanto è una piattaforma molto popolare, con ogni funzionalità utile per la gestione di questo progetto e una ottima integrazione con l'editor scelto, Visual Studio Code.

\subsection{Risorse grafiche e sonore}
All’interno del gioco Quiz me! sono presenti molti elementi grafici diversi. Questi sprite sono stati presi da OpenGameArt [14], un sito web dove è possibile trovare molte risorse grafiche con licenze che ne permettono il libero uso, da FlatIcon [15], risorsa fondamentale per trovare icone con licenza gratuita, e dove necessario sono state applicate modifiche alle immagini originali tramite il software Gimp [16], un programma gratuito completo per la modifica di immagini.

Per il progetto è stato utilizzato un font personalizzato chiamato Coolvetica, trovato sul sito web dafont.com e avente anch’esso una licenza ad uso gratuito [17].

Infine la musica di background utilizzata è stata composta appositamente per questo progetto utilizzando FLStudio [18][19].