\chapter{Fondamenti teorici}
Il presente capitolo delinea il quadro teorico e scientifico entro cui si colloca il progetto NOME,
analizzando la letteratura accademica recente riguardante i \textit{serious games}, l’efficacia dei sistemi basati su quiz e
le sfide specifiche dell’apprendimento della storia nel contesto digitale contemporaneo.

\section{Serious Games}
I videogiochi rappresentano uno dei principali mezzi di intrattenimento nell’era digitale, con un impatto significativo sulla cultura e la società moderna.
Per questo sono spesso uno dei primi strumenti utilizzati dai bambini per avvicinarsi al mondo digitale [3].

In questo contesto nascono i \textit{serious games}, ovvero videogiochi progettati con l’obiettivo di educare, oltre a fornire intrattenimento.
Grazie alla loro natura coinvolgente, aiutano i giocatori ad apprendere vari concetti in modo motivante e dinamico [4].

Negli ultimi decenni, con l'aumento della loro popolarità, i \textit{serious games} sono stati studiati da molti accademici.
La letteratura evidenzia che questi strumenti aumentano le abilità cognitive degli studenti, migliorano la loro motivazione all'apprendimento e aumentano la loro felicità [5].

\subsection{Efficacia dei Quiz nell'apprendimento}
All'interno dei \textit{serious games}, uno dei tipi molto diffusi è quello dei giochi basati su quiz.
Questo stile di gioco è stato ampiamente studiato per la sua efficacia nell'apprendimento, in particolare è stato trovato più efficace rispetto a simulazioni e avventure [6].

\section{Le sfide dell’insegnamento della Storia del ’900}
Come descritto nel Capitolo 2, lo studio della storia è ritenuto poco importante dagli studenti.
A questo è unito il problema del programma scolastico molto fitto, che porta ad non avere il tempo di approfondire gli avvenimenti storici più contemporanei, soprattutto nella seconda metà del XX secolo [7].

Per questi motivi, l'insegnamento della storia è un ambito in cui si potrebbero ottenere dei benefici significativi dall'uso di strumenti tecnologici moderni, quali i \textit{serious games}.
