\chapter{Gioco sviluppato}
Il \textit{serious game} \textcolor{red}{NOME} consiste in un programma \textit{cross-platform}, sia per computer sia per dispositivi mobili, per la gestione e l’esecuzione di quiz.

Il progetto è stato pensato per un utilizzo scolastico, focalizzato sulla storia e la grafica è stata scelta per avvicinarsi a questo tema.

L’insegnante ha accesso alla finestra di gestione dei quiz, dalla quale ha l’abilità di creare e modificare i quiz per adattarli alla parte del programma che sta coprendo in quel periodo. Successivamente può esportare i quiz creati per darli ai propri studenti. I quali, molto semplicemente, possono importarli e giocarli.

Ogni operazione viene eseguita tramite un’interfaccia grafica semplice ed intuitiva, in modo da permettere a chiunque di utilizzare il gioco senza difficoltà.

\begin{figure}[H]
    \centering
    % TEMPORANEO
    \includegraphics[width=10cm]{images/quizCRUD.png}
    \caption{Pagina iniziale del gioco.}
\end{figure}

\section{Modalità di gioco implementate}
Sono state implementate due modalità di gioco:
\begin{itemize}
    \item La modalità classica: nella quale l’utente, solitamente lo studente, può importare e giocare a dei quiz creati da un altro utente, solitamente il suo insegnante.
            
        Un quiz viene considerato completato con successo se l’utente riesce a raggiungere almeno il 60\% del punteggio massimo ottenibile.
    \item La modalità \textit{Daily}: nella quale l’utente può giocare ad un quiz giornaliero, creato automaticamente in modo da avere un quiz diverso ogni giorno.

        Se l’utente riesce a completare con successo il quiz, cioè con almeno 80\% del punteggio massimo, aumenterà di uno un contatore dei giorni consecutivi in cui il quiz giornaliero è stato completato con successo. Questo contatore viene chiamato \textit{streak}.

        La \textit{streak} viene resettata nel momento in cui il quiz giornaliero viene fallito o al primo giorno in cui non viene completato.
\end{itemize}

Entrambe le modalità di gioco si basano sullo stesso sistema di quiz, che consiste in una serie di domande a risposta multipla.
Ogni domanda ha quattro risposte possibili, di cui una o più corrette.
Il numero di domande di un quiz è variabile a discrezione dell'utente che lo crea.

\begin{figure}[H]
    \centering
    % TEMPORANEO
    \includegraphics[width=10cm]{images/quizCRUD.png}
    \caption{Pagina di una domanda durante l'esecuzione un quiz.}
\end{figure}

\section{CRUD e Import/Export dei quiz}
Per la gestione dei quiz è stata implementata una finestra di CRUD (Create, Read, Update, Delete) che permette all’utente di gestire i quiz presenti nel proprio dispositivo.

\begin{figure}[H]
    \centering
    \includegraphics[width=10cm]{images/quizCRUD.png}
    \caption{Pagina per la gestione dei quiz.}
\end{figure}
Tramite questa pagina gli insegnanti possono creare o modificare i quiz per renderli adatti ai propri studenti ed al programma che stanno insegnando in quel momento.

Nella creazione di un quiz si può scegliere liberamente il nome, il numero e il contenuto delle domande. Ogni domanda ha quattro risposte e può avere una o più risposte corrette.

La modifica di un quiz permette di modificare il numero e contenuto delle domande e le loro risposte.

Successivamente possono condividere i quiz creati, o modificati, con i propri studenti tramite le funzionalità di \textit{import/export} dei quiz.
Durante l'esportazione di un quiz, viene scelta dall'utente la cartella di destinazione.
A quel punto utilizzando strumenti intermedi di condivisione, come Microsoft Teams o Google Classroom, l'insegnante può condividere il file esportato con i propri studenti.
Gli studenti, una volta ricevuto il file, possono importarlo nel gioco tramite la funzionalità di importazione dei quiz.

Questo sistema permette di avere una gestione semplice e veloce dei quiz, senza la necessità di utilizzare sistemi complessi di gestione degli utenti o di autenticazione.

Inoltre lascia la libertà di scegliere il metodo di condivisione. Questo è un fattore molto importante in ambito scolastico, perché ogni scuola fa uso di piattaforme diverse per la condivisione dei materiali didattici e se questo gioco forzasse l'uso di una piattaforma specifica, potrebbe limitare le scuole in cui può essere utilizzato.

\begin{figure}[H]
    \centering
    \includegraphics[width=10cm]{images/quizCRUDCreate.png}
    \caption{Schermata di creazione di un quiz.}
\end{figure}

\section{Cronologia di gioco}
Ogni quiz giocato viene salvato in un file di cronologia, che tiene traccia delle partite giocate dall’utente.
Oltre al nome del quiz giocato, vengono salvati anche la data e l’ora in cui è stato giocato, il punteggio ottenuto e ogni risposta scelta, sia se corretta sia se sbagliata.

Per permettere agli utenti di tenere traccia dei propri progressi, è stata  implementata una schermata in cui è possibile visualizzare questi dati in modo leggibile.

\begin{figure}[H]
    \centering
    \includegraphics[width=10cm]{images/quizHistory.png}
    \caption{Schermata della cronologia di gioco.}
\end{figure}

Questa schermata permette di visualizzare tutte le partite giocate, ordinate in ordine cronologico, insieme al risultato ottenuto.

Cliccando sulla lente d'ingrandimento di una specifica partita, viene mostrato un riepilogo dettagliato della partita, con tutte le domande e le risposte date dall'utente, evidenziando quelle corrette e quelle sbagliate.

\begin{figure}[H]
    \centering
    \includegraphics[width=10cm]{images/quizHistoryDetails.png}
    \caption{Schermata che mostra i dettagli di un quiz giocato.}
\end{figure}


\section{Gestione dei dati}
Per la gestione dei file dei quiz è stato scelto di utilizzare il formato JSON, sia per i file dei quiz sia per i file della cronologia.

Questo formato è stato scelto in quanto è un formato leggibile sia da umani sia da computer, permettendo così ai professori che conoscono il formato di poter modificare i quiz anche senza utilizzare il gioco, mentre per gli altri docenti è stata creata una pagina specifica nel gioco.

Altri vantaggi di questo formato sono la leggerezza, che facilità le operazioni di condivisione dei quiz tra i docenti e i loro studenti, e la possibilità di usare degli strumenti già sviluppati per verificare l'integrità del file e per garantire che si tratti di un quiz valido.

\section{Meccaniche pedagogiche implementate}
Per rendere il gioco coinvolgente e divertente, sono state usate varie strategie di \textit{game design}.

In primo piano c'è il \textit{feedback} immediato dopo ogni risposta dell'utente. In questo modo lo studente capisce subito il risultato della sua scelta e può imparare dagli errori commessi.
Il feedback positivo è rappresentato da un cambio del colore del testo della risposta data in verde, mentre il feedback negativo è rappresentato da un cambio del colore del testo della risposta data in rosso e da una breve vibrazione dello stesso testo.

Anche se lo studente sbaglia una risposta, non gli viene rivelata la risposta corretta. Questo per evitare di scoraggiare l'utente e per spingerlo a cercare comunque la risposta corretta.

In base al numero di errori viene però ridotto il punteggio ottenuto. Il punteggio massimo per ogni domanda è 2 punti, se l'utente sbaglia una risposta il punteggio viene ridotto a 1 punto, se sbaglia due risposte il punteggio viene ridotto a 0 punti.
Il motivo del punteggio è per incentivare l'utente a rispondere con precisione, ma senza penalizzare troppo gli errori.
Inoltre può motivare lo studente a ripetere il quiz per migliorare il proprio punteggio e la ripetizione è un aspetto fondamentale nell'imparare nuove imformazioni.

L'ultimo aspetto pedagogico del gioco sviluppato è la \textit{Daily Challenge}. Questa modalità riprende l'idea dei giochi puzzle online, come ad esempio \textit{Wordle}[20], che propongono un puzzle diverso ogni giorno ed invitano l'utente a tornare ogni giorno grazie al mantenimento della \textit{streak}.
Questa strategia funziona perché si tratta di giochi brevi, che impegano pochi minuti al giorno, e, grazie a questo, riescono a creare una routine di apprendimento quotidiana.

Nel gioco sviluppato, la \textit{Daily Challenge} propone un quiz diverso ogni giorno e mantiene una \textit{streak} che indica il numero di giorni consecutivi in cui l'utente ha completato con successo il quiz giornaliero.
Questo incentiva gli studenti a giocare la sfida giornaliera, portando l'apprendimento al di fuori dalla classe.

Inoltre il quiz giornaliero è accessibile solo per un tentativo al giorno, aumentandone la sfida, ed è creato automaticamente, usando la data del giorno come seme per un generatore di numeri casuali. In questo modo ogni utente riceve lo stesso quiz giornaliero, permettendo così agli studenti di confrontarsi e competere tra di loro.